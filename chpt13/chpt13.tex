\begin{problem}{1} $ $

	\begin{enumerate}
		\item The characteristic polynomial for this recursive formula is
		
		\begin{equation*}
		x^2-2x+\frac{3}{4}=0,
		\end{equation*}
		which has roots 1/2 and 3/2, and therefore:		
		\begin{equation*}
		a_n=\alpha \left(\frac{3}{2} \right)^n+\beta \left(\frac{1}{2} \right)^n.
		\end{equation*}
		Using the initial conditions $a_0=1$ and $a_1=-1$ leads $\alpha=-1$ and $\beta=1$.  Thus, the solution to the recurrence equation is:
		\begin{equation*}
		a_n=-\left(\frac{3}{2} \right)^n+ \left(\frac{1}{2} \right)^n.
		\end{equation*}
		
		\item  The characteristic polynomial for this recursive formula is
		
		\begin{equation*}
		x^2-4x+4=0,
		\end{equation*}
		which can be factored into $(x-2)^2=0$.  The polynomial thus has one root, $x=2$, with a multiplicity of 2, and therefore:
		\begin{equation*}
		a_n=\alpha 2^n+\beta n 2^n.
		\end{equation*}
		Using the initial conditions $a_0=2$ and $a_1=6$ leads $\alpha=0$ and $\beta=3$.  Thus, the solution to the recurrence equation is:
		\begin{equation*}
		a_n=3n2^n.
		\end{equation*}

	\end{enumerate}

\end{problem}


\begin{problem}{2} $ $

	\begin{enumerate}
		\item Let $A_{n,k}$ be the event of observing exactly $k$ heads out of $n$ coin tosses, and let $H$ denote the event that the last coin toss is a heads.  By conditioning on the last coin toss I obtain:
		\begin{align*}
		P(A_{n, k}) &= P(A_{n, k}|H)p+P(A_{n, k}|H^c)(1-p) \\
		&= P(A_{n-1, k-1})p+P(A_{n-1, k})(1-p),
		\end{align*}
where the equality follows because if the last coin toss is heads, then we need exactly $k-1$ heads from $n-1$ toss, and if the last coin toss is tails, then we need $k$ heads from $n-1$ tosses.  Converting this to the notation used in the problem:
		\begin{equation*}
		a_{n,k}= a_{n-1, k-1}p+a_{n-1, k}(1-p)
		\end{equation*}
$\implies$
		\begin{equation*}
		a_{n+1,k+1}= a_{n, k}p+a_{n, k+1}(1-p).
		\end{equation*}

\item We recognize that this is precisely a binomial experiment, so the probability associated with exactly $k$ heads out of $n$ is given by $\binom{n}{k} p^k (1-p)^{n-k}$, and therefore:
		\begin{equation*}
		\binom{n+1}{k+1} p^{k+1} (1-p)^{(n+1)-(k+1)}= p\binom{n}{k} p^k (1-p)^{n-k}+(1-p)\binom{n}{k+1} p^{k+1} (1-p)^{n-(k+1)},
		\end{equation*}
which, when simplified results in:

		\begin{equation*}
		\binom{n+1}{k+1}= \binom{n}{k}+\binom{n}{k+1}.
		\end{equation*}
We need the restriction that $0 \le k <n$ to hold for this equation to be true, since the original recursion relation does not hold if $k = n$ (in that case the last flip cannot be a tails since we need all flips to be heads, so that the $P(A_{n, k}|H^c)$ term should be 0).




\end{enumerate}
\end{problem}










